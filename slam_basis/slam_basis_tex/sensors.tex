\chapter{传感器}

\section{视觉传感器}

结合惯性测量器件(Inertial Measurement Unit,IMU)的视觉SLAM也是现在研究热点之一。

\subsection{Monocular Camera}

\begin{enumerate}

\item 概述
\begin{enumerate}
\item 最大的问题,就是没法确切地得到深度
\item 单目SLAM只能估计一个相对深度,在相似变换空间Sim(3)中求解,而非传统的欧氏空间SE(3)。如果我们必须要在SE(3)中求解,则需要用一些外部的手段,例如GPS、IMU等传感器,确定轨迹与地图的尺度(Scale)
\item 无法确定深度同时也有一个好处:它使得单目SLAM不受环境大小的影响,因此既可以用于室内,又可以用于室外
\end{enumerate}

\item Parameters
\begin{itemize}
\item $f_x,f_y$ : focal legth
\item $c_x,c_y$ : principal point
\item $\omega$ : radial distortion, the **FOV-model**
\end{itemize}

\item Project
\begin{equation} r = \sqrt{{x_c}^2+{y_c}^2} \end{equation}
\begin{equation} r'= \frac{1}{\omega}arctan(2r tan(\frac{\omega}{2})) \end{equation}
\begin{equation}
\left[\begin{array}{c}u\\v\end{array}\right] =  
\left[\begin{array}{c}c_x\\c_y\end{array}\right] +
\left[\begin{array}{cc}f_x&0\\0&f_y\end{array}\right]  
\frac{r'}{r}
\left[\begin{array}{c}x_c\\y_c\end{array}\right]
\end{equation}

\item UnProject
\begin{equation}
\left[\begin{array}{c}{x_c}'\\{y_c}'\end{array}\right] =
\left[\begin{array}{cc}f_x&0\\0&f_y\end{array}\right]^{-1}
\Biggl(
\left[\begin{array}{c}u\\v\end{array}\right] -  
\left[\begin{array}{c}c_x\\c_y\end{array}\right]
\Biggr)
\end{equation}
\begin{equation} r' = \sqrt{ {{x_c}'}^2+{{y_c}'}^2 } \end{equation}
\begin{equation} r = \frac{tan(\omega r')}{2tan\frac{\omega}{2}} \end{equation}
\begin{equation}
\left[\begin{array}{c}x_c\\y_c\end{array}\right] =
\frac{r}{r'}
\left[\begin{array}{c}{x_c}'\\{y_c}'\end{array}\right]
\end{equation}

\item Derivative of Projection
the derivative of image frame wrt camera z=1 frame at the last computed projection: \newline
\begin{equation}
J_{projection} =
\left[\begin{array}{cc}\frac{\partial{u}}{\partial{x_c}}&\frac{\partial{u}}{\partial{y_c}}\\\frac{\partial{v}}{\partial{x_c}}&\frac{\partial{v}}{\partial{y_c}}\end{array}\right]  
\end{equation}

\end{enumerate}

\subsection{双目或多目}
\begin{enumerate}
\item 一般使用双目视觉或者三目视觉方法进行测距
\item 通过双目图像计算像素距离,是一件非常消耗计算量的事情,现在多用FPGA来完成
\item 双目相机通过多个相机之间的基线,估计空间点的位置
\end{enumerate}

\subsection{RGBD}
\begin{enumerate}
\item RGBD相机是2010年左右开始兴起的一种相机,它最大的特点是可以通过红外结构光或Time-of-Flight原理,直接测出图像中各像素离相机的距离
\item 目前常用的RGBD相机包括Kinect/Kinect V2、Xtion等
\item 出于量程的限制,主要用于室内SLAM
\end{enumerate}

\section{激光}
\begin{enumerate}
\item 激光测距单元不能够应用于水下测量
\item 优点是精度很高,速度快,计算量也不大,容易做成实时SLAM
\end{enumerate}




\section{超声波}
\begin{enumerate}
\item 在水下,由于其穿透力较强
\item 最为常用的超声波测距单元是Polaroid超声波发生器
\end{enumerate}